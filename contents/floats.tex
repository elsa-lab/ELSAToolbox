\chapter{Floats}
\label{content:floats}

Floats are containers for things in a document that cannot be broken over a page. The \LaTeX{} typesetting system by default recognizes ``table'' and ``figure'' floats. In this chapter, we first explain floating figures in detail. We then briefly introduce the adjustment of the captions of floats. Note that floating tables are mainly covered in Chapter~\ref{content:tables}.

\section{Figures}
\label{content:floats:figures}

To import external graphics, one can insert the external graphic with the command:

\begin{center}
    \verb|\includegraphics[<options>]{<file>}|.
\end{center}

The most commonly used option is \verb|width=<value>|, which scales the graphic to this value. For example, the following command imports an example image, and \LaTeX{} itself treats the image like normal text:

\begin{center}
    \verb|\includegraphics[width=.4\linewidth]{example-image}|
    \bigskip
    
    \includegraphics[width=.4\linewidth]{example-image}
\end{center}

To create a figure that floats, use the \env{figure} environment.

\bigskip

\begin{lstlisting}
\begin{figure}[placement specifier]
... figure contents ...
\end{figure}
\end{lstlisting}

The \opt{placement specifier} gives a greater degree of control over where certain floats are placed. We list the commonly used specifiers as follows.

\bigskip

\begin{tabularx}{.9\linewidth}{cX}
    \toprule
    Specifier & Permission \\
    \midrule
    h & Place the float here, i.e., approximately at the same point it occurs in the source text (however, not exactly at the spot) \\
    t & Position at the top of the page. \\
    b & Position at the bottom of the page. \\
    \bottomrule
\end{tabularx}

\bigskip

The following example creates Fig.~\ref{fig:example-a} that placed at the the top of the page:

\bigskip

\lstinputlisting{figures/example-a.tex}
\begin{figure}[t]
    \centering
    \includegraphics[width=.4\linewidth]{example-image-a}
    \caption{An example image with an A.}
    \label{fig:example-a}
\end{figure}

Instead of placing floats on the current page, one may consider moving floats to the next page. To achieve this, surround the float with \verb|\afterpage{<float>}| to make it appear after the current page. For example, the following commands create Fig.~\ref{fig:example-b} that placed at the top of the following page:

\bigskip

\lstinputlisting{figures/example-b.tex}
\afterpage{
\begin{figure}[t]
    \centering
    \includegraphics[width=.4\linewidth]{example-image-b}
    \caption{An example image with a B.}
    \label{fig:example-b}
\end{figure}
}

\section{Subfloats}
\label{content:floats:subfloats}

In this section, we introduce the \pkg{subcaption} package that allows users to have subfloats within floats. Please note that \elsatoolbox{} loads \pkg{subcaption} that may be incompatible with some template files. To solve this issue, the \pkg{subfig} package is a useful alternative to \pkg{subcaption}. Before using \pkg{subfig}, users should disable \pkg{subcaption} by adding the option \opt{nosubcap} to \elsatoolbox{} as the following:

\begin{center} 
    \verb|\usepackage[nosubcap]{elsatoolbox}|.
\end{center}

The following example creates Fig.~\ref{fig:exmples-a-b}.

\bigskip

\lstinputlisting{figures/exmples-a-b.tex}
\begin{figure}[t]
\centering
\begin{subfigure}[t]{.33\linewidth}
    \centering
    \includegraphics[width=.9\linewidth]{example-image}
    \subcaption{\texttt{example-image}.}
    \label{fig:exmples-a-b:example-image}
\end{subfigure}% <-this % stops a space
\begin{subfigure}[t]{.33\linewidth}
    \centering
    \includegraphics[width=.9\linewidth]{example-image-a}
    \subcaption{\texttt{example-image-a}.}
    \label{fig:exmples-a-b:example-image-a}
\end{subfigure}% <-this % stops a space
\begin{subfigure}[t]{.33\linewidth}
    \centering
    \includegraphics[width=.9\linewidth]{example-image-b}
    \subcaption{\texttt{example-image-b}.}
    \label{fig:exmples-a-b:example-image-b}
\end{subfigure}
\caption{Example subfigures using \texttt{subcaption}.}
\label{fig:exmples-a-b}
\end{figure}


\section{Captions}
\label{content:floats:captions}

Although adjusting the vertical spacing between floats and captions to compact/relax the main text is not recommended, it is wiser to use smaller floats or reduce words. We still provide ways to customize captions using \pkg{subcaption} as following:

\bigskip

\begin{lstlisting}
% Customize figure environment
\captionsetup[figure]{
    font=small, % Font size option
    skip=1ex    % Vertical space between float and caption
}

% Customize table environment
\captionsetup[table]{
    font=small, % Font size option
    skip=1ex    % Vertical space between float and caption
}

% Customize subfigure environment
\captionsetup[sub]{
    font=small, % Font size option
    skip=1ex    % Vertical space between subfloat and subcaption
}
\end{lstlisting}